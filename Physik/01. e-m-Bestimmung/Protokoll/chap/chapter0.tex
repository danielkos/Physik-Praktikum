\chapter{Einleitung}
\label{Einleitung}
In dem durchgeführten Versuch wird mittels eines Fadenstrahlrohres und der Methode von Busch die spezifische Elektronenladung bestimmt. 

Bei der Bestimmung mittels Fadenstrahlrohr (siehe Kapitel \ref{HemholtzspulenpaarUntersuchung}) wird ausgenutzt, dass Elektronen im Magnetfeld aufgrund der Lorentzkraft eine Kreisbahn bilden und bei Kollision mit dem Wasserstoffgasmolekülen Photonen abgeben, welche wiederum als leuchtender Elektronenstrahl sichtbar werden [ref].

Hingegen wird bei der Methode von Busch (siehe Kapitel \ref{MethodeBusch}) eine Braun'sche Röhre verwendet, bei welcher die Elektronen von der Kathode zur Anodode beschleunigt und auf einem Leuchtschirm auftreffen. Mit geeigneter Beschleunigungsspannung und Spulenstrom kann die spezifische Elektronenladung bestimmt werden.

Mit Hilfe der beiden Methoden wird ein Wert für die spezifische Elektronenladung ermittelt, mit dem Literaturwert($1.759*10^{11}\frac{C}{kg}$) verglichen und potentielle Fehlerquelle diskutiert.
%% https://lp.uni-goettingen.de/get/text/1545