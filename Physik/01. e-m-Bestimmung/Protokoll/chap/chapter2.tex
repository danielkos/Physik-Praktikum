\chapter{e/m-Bestimmung nach Methode von Busch}
\label{MethodeBusch}
\section{Vorbereitung des Versuchs}
Entsprechend Aufgabenbeschreibung stellten wir die Ablenkspannung und  Deflektorspannung so ein, dass wir einen maximal langen Strich erhielten.So könnten wir beobachten, dass bei Änderung der Ablenkspannung XX passiert und bei der Deflektorspannung YY. Beim Einstellen des maximal langen Strichs stellten wir fest, dass der Strich mittig unterbrochen schien bzw. weniger intensiv war. Ebenso war es schwierig mit zuvor eingestellten Messwerten einen exakten Punkt zu erzielen. Nach einigen Justierungen konnten wir einen möglichst kleinen Punkt auf dem Schirm der Kathodenstrahlen erzielen.
\section{Messung des nötigen Spulenstroms für die Beschleunigungsspannung}
Gemäß Aufgabe stellten wir die Beschleunigungsspannung auf Werte zwischen 500V und 700V. Hierfür wählten wir eine Schrittweite von 50V und führten den nötigen Spulenstrom zu, welcher nötig war um für die unterschiedlichen Beschleunigungsspannungen einen Punkt auf den Schirm zu erzielen. Die nötigen Spulenströme für die jeweiligen Beschleunigungspannungen sind in Abbildung enthalten.

