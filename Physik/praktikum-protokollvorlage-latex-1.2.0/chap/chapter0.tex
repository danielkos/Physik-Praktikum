\chapter{Einleitung}
In dem durchgeführten Versuch wird mittels des Fadenstrahlrohres und der Methode von Busch spezifische Elektronenladung bestimmt. Bei der Bestimmung mittels des Fadenstrahlrohrs wird ausgenutzt, dass Elektronen im Kugelinneren aufgrund der herschenden Lorentzkraft eine Kreisbahn bilden und bei Kollision mit dem Wasserstofgasmolekülen Energie in Form von Photonen abgeben, welche wiederrum als leuchtender Elektronenstrahl sichtbar werden [ref].

Hingegen wird bei der Methode von Busch eine Braun'sche Röhre verwendet, bei welcher die Elektronen von der Kathode zur Anodode beschleunigt und auf einem Leuchtschirm abgebildet werden. Womit bei geeigneter Beschleunungsspannung und Stärke des Magnetfelds die spezifische Elektornenladung bestimmt werden kann.

Anhand der Durchführung, Vergleich mit dem Literaturwert von XX [ref] und zugehörigren Interpretation der Ergebnissen werden die jeweiligen Vor- und Nachteile der verschiedenen Methoden diskutiert.
