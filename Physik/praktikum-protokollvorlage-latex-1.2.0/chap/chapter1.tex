\chapter{e/m-Bestimmung mit dem Fadenstrahlrohr}
\section{Untersuchung des Feldes eines Helmholtzspulenpaars}
Um die Hallspannung zu bestimmen, bauten wir die zusätzliche Helmholtzspule mit Messplatte sowie die Messplatte entsprechend der Aufgabenbeschreibung auf. Anschließend führten wir einen Nullabgleich der Hallsonde durch um den vorherschenden geomagnetischen Feldern als auch anderen störenden Einflüssen während der Messung entgegen zu wirken. Hier stellten wir bereits fest, dass kleinste Berührungen an den Messgeräten, Kabeln und anderen Anschlüssen den Nullabgleich leicht veränderten, weshalb wir bei jedem Umbau während der Abarbeitung der einzelnen Versuche,  vor jedem neuen Versuch erneut einen Nullabgleich durchführten.

Entsprechend anweisung seitens des Betreuers als auch den Hinweis auf dem Aufgabenblatt messten wir den ersten und höchsten Ausschlag  der Hallspannung für die jeweilige Spulenströme von 1A, 1.5A und 2A sowie den vorgegebenen Position der Messplatte. Aufgrund zuvor erwähnten Problemen seites des sich ändernden Nullabgleichs, führten wir die Messung für 1A erneut durch, da diese anfangs uns falsch erschienen. Wir stellten ebenso fest, dass die Amperzahl nicht konstant auf dem Ampermeter angezeigt wurde und diese scheinbar wärend es Versuchs abnahm. Dies ist vermutlich auf die mangelnde Genauigkeit des verwendeten Messgerät szurück zu führen, denn die Anzeige der Stromquelle blieb konstant.

(VLLT diese messfehler feststellung zuvor erwähnen, oder irgendwie einbringen dass man diese nun nicht mehr erwähnt.)


DISKUSSION ERGEBNISSE
\section{Kalibrieren der Hallsonde}
Um die Hallsonde möglichst genau zu Eichen wurde sie in die Mitte der Spule eingeführt und 10 Messungen der Spannung durchgeführt. 

Entsprechend konnten wir das Magnetfeld der Spule über folgende Formel berechnen.



Die verwendete Spule hatte folgende Maßzahlen:
L = mm
r = mm
n = Windungen.
 Für die magnetische Feldkonstante $\mu_0$ wurde der Literaturwert von BLABLA verwendet [ref]. 
 
 Die berechneten Werte des Magnetfelds für die jeweiligen Stromsträkrnen und Hallspannungensind in XX und YY zu finden. In YY sieht man, dass die Regressionsgerade in etwa den gemessenen Werten entspricht.
 Um nun für Aufgabe 1.3 den gemessenen mit dem in der Aufgabe angebenen Soll-Magnetfeldwert zu vergleichen, mussten wir den Steigungsfaktor des Magnetfelds bestimmten. Hierfür nutzen wir die Beziehung aus, dass
 Fl=Fel => evB=eE, => v*B=U.h/d
 somit ergibt isch für B = 1/vd*Uh. Da sowohl v als auch d von der Spannung abhängige Konstante ist, entspricht diese der Steigung der Geraden, womit wir B=m*uh erhalten mit m = 1/vd.
 
 Entsprechend erhalten wir für unsere Messwerte:
 
 m = XXX
\section{Vergleich zwischen gemessenen und berechneten Wert des Mittenfeldes zwischen den Helmholtzspulen}
Um die Genauigkeit unserer Messwerte und dem daraus berechneten Magnetfeld zu überprüfen haben wir für die drei Stromstärken 1A,1.5A, und 2A unseren gemessenen als auch den Soll-Wert vergleichen. Die Abweichungen vom Sollwert sind in Prozent angegeben.
 
 In XX sieht man, dass aufgrund der zuvor ebschriebenen Messunsicherheiten seitens der Messgeräte (siehe Kapitel 1,111) sowie den sich fortpflanzenden Fehlern wärend der Nullabstimmung als auch der in 1.2 beschrieben Eichung, Abweichungen auftreten. So erhielten wir dennoch Abweichungen die noch sehr gering sind.
 
 Im Nachfolgenden wird nun mit dem Sollwert weiter gerechnet um die Fortpflanzung der zuvor beschrieben Messunsicherheiten zu vermeiden.
 
 
\section{Messung des Durchmessers der Elektronenkreisbahn im Fadenstrahlrohr}
\paragraph{In Abhängigkeit der Anodenspannung}
Entsprechend der Aufgabengespreibung bauten wir den Versuch auf und bestimmten für Anodenspannungen von 125V bis 250V bei jeweils 1A und 2A die zugehörigen Durchmesser der Elektronenkreisbahn. Hierfür wählten wir einen adjazenten Abstand von 25V. Parallaxenfehler bei der Bestimmung des Durchmessers wurden gemäß Anordung möglichst kleingehalten. Jedoch stellte sich das exakte bestimmen des Durchmessers als dennoch schwierig heraus, was an dem stelleweise diffusen Elektronenstrahl zuzschrieben ist. Um möglichst gute Ergebnisse zu erzielen überprüften wir jeweils die Bestimmung des Durchmessers des jeweils anderen. Aufgrund der zu großen Kreisbahn un dem damit überschrittenen Messbereich unserer Versuchsanordnung konnten wir bei 1A für Anodenspannungen größer als 200V keinen Durchmesser der Kreisbahn bestimmen. Hingegen konnten wir bei 2A für alle Anodenspannungen einen Durchmesser bestimmen. So ermittelten wir die in Abbildung XX gezeigten Messwerte. 




\paragraph{In Abhängigkeit des Spulenstroms}
Nun untersuchten wir entsprechend der Aufgabenbeschreibung in b die  Durchmesser der Elektronenkreisbahn für zwei feste Beschleunigungsspannungen (150V und 250V) für Spulenströme zwischen 1A bis 2A. Hierführ wählten den adjazenten Abstand von 0.2A. Auch hier trat das soeben beschriebene Problem des Diffusen Elektronenstrahls auf. Bis auf den Messwert für 1A bei 250V konnten wir für jede Konfiguration einen Durchmesser bestimmen, welche in Abbildung XX zu sehen sind.






